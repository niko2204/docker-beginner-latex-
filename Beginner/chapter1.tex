\chapter{소개}
\section{Docker란 무엇인가요?}
Wikipedia는 \href{https://www.docker.com/}{Docker}를 다음과 같이 정의합니다:
\begin{quote}
    Docker is a set of platform as a service (PaaS) products that use OS-level virtualization to deliver software in packages called containers. The service has both free and premium tiers. The software that hosts the containers is called Docker Engine. It was first released in 2013 and is developed by Docker, Inc.

    Docker is a tool that is used to automate the deployment of applications in lightweight containers so that applications can work efficiently in different environments in isolation.

    Docker는 OS 수준 가상화를 사용하여 컨테이너라고 불리는 패키지 형태로 소프트웨어를 제공하는 플랫폼 서비스(PaaS) 제품군입니다. 이 서비스는 무료 및 프리미엄 등급을 모두 제공합니다. 컨테이너를 호스팅하는 소프트웨어는 Docker Engine이라고 불리며, 2013년에 처음 출시되었고 Docker, Inc.에서 개발했습니다.
    
    Docker는 애플리케이션을 가벼운 컨테이너에 배포하는 작업을 자동화하는 도구로, 애플리케이션이 서로 독립된 다양한 환경에서 효율적으로 작동할 수 있도록 합니다.
\end{quote}

Docker는 개발자, 시스템 관리자 등이 응용 프로그램을 호스트 운영 체제(즉, 리눅스)에서 실행할 수 있는 샌드박스(컨테이너라고 함)에 쉽게 배포할 수 있게 하는 도구입니다. Docker의 주요 이점은 응용 프로그램과 모든 종속성을 소프트웨어 개발을 위한 표준화된 단위로 패키징할 수 있다는 것입니다. 가상 머신과 달리 컨테이너는 높은 오버헤드가 없어서 기본 시스템 및 리소스를 보다 효율적으로 사용할 수 있습니다.

\section{컨테이너란 무엇인가요?}
오늘날 업계 표준은 소프트웨어 응용 프로그램을 실행하기 위해 가상 머신(VM)을 사용하는 것입니다. VM은 가상 하드웨어에서 실행되는 게스트 운영 체제 내부에서 응용 프로그램을 실행합니다.

VM은 응용 프로그램의 전체 프로세스를 격리하는 데 매우 유용합니다. 호스트 운영 체제의 문제가 게스트 운영 체제에서 실행되는 소프트웨어에 영향을 미칠 수 있는 방법은 거의 없습니다. 그러나 이러한 격리는 큰 비용이 듭니다. 게스트 운영 체제가 사용할 하드웨어를 가상화하는 데 소요되는 계산 오버헤드는 상당합니다.

컨테이너는 호스트 운영 체제의 저수준 메커니즘을 활용하여 가상 머신의 대부분의 격리를 적은 계산 능력으로 제공합니다.

\section{컨테이너를 왜 사용하나요?}
컨테이너는 응용 프로그램이 실제로 실행되는 환경에서 추상화될 수 있는 논리적 패키징 메커니즘을 제공합니다. 이 디커플링 덕분에 컨테이너 기반 응용 프로그램은 개인 데이터 센터, 공용 클라우드 또는 개발자의 개인 랩톱 등 대상 환경에 관계없이 쉽게 배포할 수 있습니다. 이는 개발자가 다른 응용 프로그램과 격리된 예측 가능한 환경을 만들고 어디에서나 실행할 수 있게 합니다.

운영 관점에서 볼 때, 이동성 외에도 컨테이너는 리소스를 보다 세밀하게 제어할 수 있어 인프라의 효율성을 향상시킬 수 있으며, 이는 컴퓨팅 리소스의 더 나은 활용으로 이어질 수 있습니다.

\section{이 튜토리얼은 무엇을 가르쳐줄까요?}
이 튜토리얼은 Docker를 실제로 다루는 데 필요한 모든 것을 알려주는 종합적인 가이드입니다. Docker 환경을 이해하는 것 외에도, 클라우드에서 자신의 웹앱을 구축하고 배포하는 실습 경험을 제공할 것입니다. 우리는 Amazon Web Services를 사용하여 정적 웹사이트와 두 개의 동적 웹앱을 EC2를 사용하여 Elastic Beanstalk과 Elastic Container Service에 배포할 것입니다. 배포에 대한 사전 경험이 없더라도 이 튜토리얼은 시작하는 데 필요한 모든 것을 제공합니다.