\chapter{시작하기}
이 문서는 Docker의 특정 측면을 설명하는 여러 섹션으로 구성되어 있습니다. 각 섹션에서는 명령을 입력하거나 코드를 작성할 것입니다. 튜토리얼에서 사용되는 모든 코드는 Github 저장소에 있습니다.
\begin{quote}
참고: 이 튜토리얼은 Docker 버전 27.1.1, build 6312585를 사용합니다. 튜토리얼의 일부가 향후 버전과 호환되지 않는 경우, 문제를 제기해 주세요. 감사합니다!
\end{quote}

\section{사전 요구 사항}
이 튜토리얼을 위한 특정 기술 요구 사항은 없습니다. 커맨드 라인 사용과 텍스트 편집기 사용에 대한 기본적인 이해만 있으면 됩니다. 이 튜토리얼은 git clone을 사용하여 로컬 저장소를 복제합니다. 시스템에 Git이 설치되어 있지 않으면 설치하거나 Github에서 수동으로 zip 파일을 다운로드해야 합니다. 웹 응용 프로그램 개발에 대한 사전 경험이 도움이 되지만 필수는 아닙니다. 튜토리얼을 진행하면서 몇 가지 클라우드 서비스를 사용할 것입니다. 함께 따라하고 싶다면 다음 웹사이트에 계정을 만들어 주세요:
\begin{itemize}
    \item \href{http://aws.amazon.com/}{Amazon Web Services}
    \item \href{https://virtualbox.org/}{VirtualBox}
    \item \href{https://docs.getutm.app/installation/macos/}{UTM}
    \item \href{https://hub.docker.com/}{Docker Hub}
\end{itemize}

\section{apt 저장소를 이용한 설치}

새 호스트 머신에 처음으로 Docker Engine을 설치하기 전에, Docker 저장소를 설정해야 합니다. 이후에는 저장소에서 Docker를 설치하고 업데이트할 수 있습니다.

컴퓨터에 모든 도구를 설정하는 것은 어려울 수 있지만, 다행히도 Docker가 안정화됨에 따라 Docker를 좋아하는 OS에서 실행하는 것이 매우 쉬워졌습니다.

몇 릴리즈 전까지는 OSX와 Windows에서 Docker를 실행하는 것이 꽤 번거로웠습니다. 그러나 최근 Docker는 이러한 OS의 사용자를 위한 온보딩 경험을 크게 개선하여 이제 Docker를 실행하는 것이 매우 간단합니다. Docker의 시작하기 가이드는 Mac, Linux 및 Windows에서 Docker를 설정하는 자세한 지침을 제공합니다.

각각 설치 방법은 아래 링크를 클릭 하세요. 설치 파일ㅇ르 다운로드 받거나 명령어를 복사하여 터미널에 붙여넣기 하세요.:
\begin{itemize}
    \item \href{https://docs.docker.com/docker-for-mac/}{Docker for Mac}
    \item \href{https://docs.docker.com/engine/install/ubuntu/}{Docker for Ubuntu}
    \item \href{https://docs.docker.com/desktop/install/windows-install/}{Docker for Windows}
\end{itemize}

\subsection{Docker의 apt 저장소 설정}

\begin{enumerate}
\item Docker의 공식 GPG 키 추가:
\begin{lstlisting}[language=Shell]
# Add Docker's official GPG key:
sudo apt-get update
sudo apt-get install ca-certificates curl
sudo install -m 0755 -d /etc/apt/keyrings
sudo curl -fsSL https://download.docker.com/linux/ubuntu/gpg -o /etc/apt/keyrings/docker.asc
sudo chmod a+r /etc/apt/keyrings/docker.asc
\end{lstlisting}

\item 저장소를 apt 소스에 추가:
\begin{lstlisting}[language=bash]
# Add the repository to Apt sources:
echo \
    "deb [arch=$(dpkg --print-architecture) signed-by=/etc/apt/keyrings/docker.asc] https://download.docker.com/linux/ubuntu \
    $(. /etc/os-release && echo "$VERSION_CODENAME") stable" | \
    sudo tee /etc/apt/sources.list.d/docker.list > /dev/null
sudo apt-get update
\end{lstlisting}
\end{enumerate}

\textbf{주의}: Linux Mint와 같은 Ubuntu 파생 배포판을 사용하는 경우, \texttt{VERSION\_CODENAME} 대신 \texttt{UBUNTU\_CODENAME}을 사용해야 할 수 있습니다.

\subsection{Docker 패키지 설치}

\subsubsection{최신 버전 설치}
최신 버전을 설치하려면, 다음 명령어를 실행하십시오:
\begin{lstlisting}[language=bash]
 sudo apt-get install docker-ce docker-ce-cli containerd.io docker-buildx-plugin docker-compose-plugin
\end{lstlisting}

\subsection{설치 확인}

\texttt{hello-world} 이미지를 실행하여 Docker Engine 설치가 성공적으로 완료되었는지 확인할 수 있습니다:
\begin{lstlisting}[language=bash]
 sudo docker run hello-world
\end{lstlisting}
이 명령어는 테스트 이미지를 다운로드하여 컨테이너에서 실행합니다. 컨테이너가 실행되면 확인 메시지를 출력하고 종료됩니다.

이제 Docker Engine을 성공적으로 설치하고 시작하였습니다.

\begin{lstlisting}[language=bash]
$ docker run hello-world

Hello from Docker!
This message shows that your installation appears to be working correctly.

To generate this message, Docker took the following steps:
 1. The Docker client contacted the Docker daemon.
 2. The Docker daemon pulled the "hello-world" image from the Docker Hub.
    (arm64v8)
 3. The Docker daemon created a new container from that image which runs the
    executable that produces the output you are currently reading.
 4. The Docker daemon streamed that output to the Docker client, which sent it
    to your terminal.

To try something more ambitious, you can run an Ubuntu container with:
 $ docker run -it ubuntu bash

Share images, automate workflows, and more with a free Docker ID:
 https://hub.docker.com/

For more examples and ideas, visit:
 https://docs.docker.com/get-started/

\end{lstlisting}

\begin{lstlisting}[language=bash]
$ docker -v
Docker version 27.1.2, build 6312585
\end{lstlisting}

